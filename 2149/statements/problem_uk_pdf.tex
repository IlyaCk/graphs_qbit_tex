\begin{problem}{}{}{}{0.2 секунди}{64 мегабайти}
% Кратні ребра в неорієнтованому графі

Неорієнтований граф задано списком ребер. Перевірте, чи містить він кратні (інша назва $-$ паралельні) ребра.

\InputFile
У першому рядку вхідних даних задано число $K$ ($1 \le K \le 10$) $-$ кількість тестованих графів. 
Далі йде $K$ блоків, кожен з яких описує один неорієнтований граф списком ребер. 
Для кожного блоку спочатку задано числа $N$ ($1 \le N \le 100$) $-$ кількість вершин і $M$ ($1 \le M \le 10\,000$) $-$
кількість ребер, а потім слідує $M$ рядків по $2$ числа $-$ опис ребер неорієнтованого графа.

\OutputFile
Для кожного графа, що тестується, програма повинна в окремий рядок вивести <<{\t{YES}}>>, якщо цей граф
містить кратні ребра і <<{\t{NO}}>> $-$ в іншому випадку.

\Example

\begin{example}
\exmp{2
3 3 
1 2 
2 3 
1 3
3 3 
1 2 
2 3 
2 1
}{NO
YES
}%
\end{example}

\end{problem}

