\begin{problem}{}{}{}{0.2 секунды}{64 мегабайта}
% Степени вершин по матрице смежности

Неориентированный граф задан матрицей смежности. Найдите степени всех вершин графа.

\InputFile
В первой строке входных данных задано число $N$ ($1 \le N \le 100$) $-$ количество вершин в графе.
Далее следует $N$ строк по $N$ чисел $-$ матрица смежности графа $G$. В матрице смежности элемент $G_{i,j}=1$, если
существует ребро, соединяющее вершины $i$ и $j$, $G_{i,j}=0$ $-$ в противном случае.

\OutputFile
Ваша программа должна вывести $N$ чисел $-$ степени всех вершин графа.

\Example
\begin{example}
\exmp{4
0 1 0 1
1 0 1 1
0 1 0 0
1 1 0 0
}{2 3 1 2
}%
\end{example}

\end{problem}

