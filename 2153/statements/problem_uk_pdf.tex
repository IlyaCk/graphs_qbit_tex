\begin{problem}{}{}{}{0.2 секунди}{64 мегабайти}
% Напівстепені вершин за матрицею суміжності

Орієнтований граф задано матрицею суміжності. Знайдіть напівстепені заходу і напівстепені виходу всіх вершин графа.
Тобто для кожної вершини треба порахувати скільки в неї входить ребер і скільки з неї виходить ребер.

\InputFile
У першому рядку вхідних даних задано число $N$ ($1 \le N \le 100$) $-$ кількість вершин у графі.
Далі йде $N$ рядків по $N$ чисел $-$ матриця суміжності графа $G$. У матриці суміжності елемент $G_{i,j}=1$, якщо
існує ребро, що з'єднує вершини $i$ та $j$, $G_{i,j}=0$ $-$ в іншому випадку.

\OutputFile
Ваша програма повинна вивести $N$ пар чисел (кожна пара в окремому рядку). Для кожної вершини спочатку виведіть 
напівстепені заходу і потім напівстепені виходу.

\Example

\begin{example}
\exmp{4 
0 1 0 1 
1 0 1 1 
0 1 0 0 
1 1 1 1
}{2 2 
3 3 
2 1 
3 4
}%
\end{example}

\end{problem}

