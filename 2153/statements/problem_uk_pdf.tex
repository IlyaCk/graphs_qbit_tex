\begin{problem}{}{}{}{0.2 секунды}{64 мегабайта}
% Полустепени вершин по матрице смежности

Ориентированный граф задан матрицей смежности. Найдите полустепени захода и полустепени исхода всех вершин графа.
Т.е. для каждой вершины надо посчитать сколько в неё входит рёбер и сколько из неё выходит рёбер.

\InputFile
В первой строке входных данных задано число $N$ ($1 \le N \le 100$) $-$ количество вершин в графе.
Далее следует $N$ строк по $N$ чисел $-$ матрица смежности графа $G$. В матрице смежности элемент $G_{i,j}=1$, если
существует ребро, соединяющее вершины $i$ и $j$, $G_{i,j}=0$ $-$ в противном случае.

\OutputFile
Ваша программа должна вывести $N$ пар чисел (каждая пара в отдельной строке). Для каждой вершины сначала выведите 
полустепень захода и затем полустепень исхода.

\Example

\begin{example}
\exmp{4 
0 1 0 1 
1 0 1 1 
0 1 0 0 
1 1 1 1
}{2 2 
3 3 
2 1 
3 4
}%
\end{example}

\end{problem}

