\begin{problem}{}{}{}{0.2 секунди}{64 мегабайти}
% Напівступені вершин за списком ребер

Орієнтований граф задано списком ребер. Знайдіть напівстепені заходу і напівстепені виходу всіх вершин графа.
Тобто для кожної вершини треба порахувати скільки в неї входить ребер і скільки з неї виходить ребер.

\InputFile
У першому рядку вхідних даних задано числа $N$ ($1 \le N \le 100$) $-$ кількість вершин у графі та 
$M$ ($1 \le M \le N\cdot (N-1)/2$) $-$ кількість ребер.
Далі йде $M$ рядків по $2$ числа в кожному $-$ список ребер графа. 

\OutputFile
Ваша програма має вивести $N$ пар чисел (кожна пара в окремому рядку). Для кожної вершини спочатку виведіть 
напівстепінь заходу і потім напівстепінь результату.

\Example

\begin{example}
\exmp{4 4 
1 2 
1 3 
2 3 
3 4 
}{0 2 
1 1 
2 1 
1 0
}%
\end{example}

\end{problem}

