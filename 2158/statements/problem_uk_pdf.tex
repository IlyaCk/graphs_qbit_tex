\begin{problem}{}{}{}{0.2 секунди}{64 мегабайти}
% Напівповний граф

Орієнтований граф називається напівповним, якщо між будь-якою парою його різних вершин є хоча б одне ребро. 
Для заданого списком ребер орієнтованого графа перевірте, чи є він напівповним.

\InputFile
У першому рядку вхідних даних записано число $K$ ($1 \le K \le 10$) $-$ кількість тестованих графів. 
Далі йде $K$ блоків, кожен з яких описує один орієнтований граф списком ребер. 
Для кожного блоку спочатку задано числа $N$ ($1 \le N \le 100$) $-$ кількість вершин і $M$ ($1 \le M \le 10\,000$) $-$
кількість ребер, а потім йде $M$ рядків по $2$ числа $-$ опис ребер орієнтованого графа. У графі можуть бути кратні ребра
і петлі.

\OutputFile
Для кожного графа, що тестується, програма повинна вивести окремим рядком <<{\t{YES}}>>, якщо цей граф
є напівповним, чи <<{\t{NO}}>> $-$ в іншому випадку.

\Example

\begin{example}
\exmp{2
3 3 
1 2 
1 3 
2 3
3 2 
1 2 
2 3
}{YES
NO
}%
\end{example}

\end{problem}

