\begin{problem}{}{}{}{0.2 секунда}{64 мегабайта}
% Транзитивность неориентированнго графа

Граф называется транзитивным, если всегда из того, что вершины $u$ и $v$ соединены ребром и вершины $v$ и $w$ соединены 
ребром следует, что вершины $u$ и $w$ соединены ребром. Проверьте, что заданный неориентированный граф является транзитивным.

\InputFile
В первой строке входных данных записано число $K$ ($1 \le K \le 10$) $-$ количество тестируемых графов. 
Далее следует $K$ блоков, каждый из которых описывает один неориентированный граф списком рёбер. 
Для каждого блока сначала заданы числа $N$ ($1 \le N \le 100$) $-$ количество вершин и $M$ ($1 \le M \le 10\,000$) $-$
количество рёбер, а затем следует $M$ строк по $2$ числа $-$ описание рёбер неориентированного графа.

\OutputFile
Для каждого тестируемого графа программа должна в отдельную строку вывести <<{\t{YES}}>>, если этот граф
является транзитивным и <<{\t{NO}}>> $-$ в противном случае.

\Example

\begin{example}
\exmp{2
3 3 
1 2 
1 3 
3 2
3 2 
1 2 
1 3
}{YES
NO
}%
\end{example}

\end{problem}

