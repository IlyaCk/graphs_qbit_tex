\begin{problem}{}{}{}{0.2 секунди}{64 мегабайти}

Граф називається транзитивним, якщо з того, що вершини $u$ і $v$ та $v$ і $w$ попарно з'єднані ребрами, завжди випливає, що вершини $u$ і $w$ з'єднані ребрами.
Перевірте, чи є заданий неорієнтований граф транзитивним.

\InputFile
У першому рядку вхідних даних записано число $K$ ($1 \le K \le 10$) $-$ кількість тестів. 
Далі йде $K$ блоків, кожен з яких описує один неорієнтований граф списком ребер. 
Для кожного блоку спочатку задано числа $N$ ($1 \le N \le 100$) $-$ кількість вершин і $M$ ($1 \le M \le 10\,000$) $-$ кількість ребер,
а потім йде $M$ рядків по $2$ числа $-$ опис ребер неорієнтованого графа.

\{OutputFile
Для кожного протестованого графа програма повинна вивести <<{\t{YES}}>> в окремому рядку, якщо цей граф є транзитивним і <<{\t{NO}}>>$-$ у протилежному випадку.

\Example

\begin{example}
\exmp{2
3 3 
1 2 
1 3 
3 2
3 2 
1 2 
1 3
YES
NO
}%
\end{example}

\end{problem}
