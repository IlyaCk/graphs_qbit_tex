\begin{problem}{}{}{}{0.2 секунды}{64 мегабайта}
% Транзитивность ориентированного графа

Ориентированный граф называется транзитивным, если для любых трех различных вершин $u$, $v$ и $w$ из того, что из $u$ 
в вершину $v$ ведет ребро и из вершины $v$ в вершину $w$ ведет ребро, следует, что из вершины $u$ в вершину $w$ ведет ребро.
Проверьте, что заданный ориентированный граф является транзитивным.

\InputFile
В первой строке входных данных записано число $K$ ($1 \le K \le 10$) $-$ количество тестируемых графов. Далее следует $K$
блоков, каждый из которых описывает одну матрицу смежности ориентированного графа. 
Для каждого блока сначала задано число $N$ ($1 \le N \le 100$) $-$ 
размерность матрицы, а затем следует описание самой матрицы $-$ $N$ строк по $N$ чисел, каждое их которых 0 или 1.

\OutputFile
Для каждого тестируемого графа программа должна в отдельную строку вывести <<{\t{YES}}>>, если этот граф
является транзитивным и <<{\t{NO}}>> $-$ в противном случае.

\Example

\begin{example}
\exmp{2
3 
0 1 1
0 0 1 
0 0 0
3 
0 1 1 
1 0 0 
0 1 0 
}{YES
NO
}%
\end{example}

\end{problem}

