\begin{problem}{}{}{}{}{0.2 секунди}{64 мегабайти}

Орієнтований граф називається транзитивним, якщо для будь-яких трьох різних вершин $u$, $v$ і $w$ з того, що з $u$ у вершину $v$ веде ребро і з вершини $v$ у вершину $w$ веде ребро, випливає, що з вершини $u$ у вершину $w$ веде ребро.
Перевірте, що заданий орієнтований граф є транзитивним.

\InputFile
У першому рядку вхідних даних записано число $K$ ($1 \le K \le 10$) $-$ кількість тестованих графів.
Далі йде $K$ блоків, кожен з яких описує одну матрицю суміжності орієнтованого графа.
Для кожного блоку спочатку задано число $N$ ($1 \le N \le 100$) $-$ розмірність матриці, а потім йде опис самої матриці $-$ $N$ рядків по $N$ чисел, кожне з яких 0 або 1.

\OutputFile
Для кожного графа, що тестується, програма повинна в окремий рядок вивести <<{\t{YES}}}>>, якщо цей граф є транзитивним і <<{\t{NO}}}>> $-$ в іншому випадку.

\Example

\begin{example}
\exmp{2
3 
0 1 1
0 0 1 
0 0 0
3 
0 1 1 
1 0 0 
0 1 0 
}{YES
NO
}
\end{example}

\end{problem}
