\begin{problem}{}{}{}{0.2 секунды}{64 мегабайта}
% Матрица смежности взвешенного графа

Неориентированный взвешенный граф задан матрицей смежности. Найдите в этом графе все рёбра с максимальным и минимальным
весом.

\InputFile
В первой стоке входных данных задано число $N$ ($1 \le N \le 100$) $-$ количество вершин в графе.
Далее следует $N$ строк по $N$ чисел $-$ матрица смежности графа $G$. В матрице смежности элемент $G_{i,j} \ne 0$, если
существует ребро, соединяющее вершины $i$ и $j$, $G_{i,j}=0$ $-$ в противном случае. Элемент $G_{i,j}$ в матрице
смежности означает вес ребра, соединяющего вершины с номерами $i$ и $j$.

\OutputFile
Выведите сначала все рёбра, имеющие максимальный вес, а затем все рёбра имеющие минимальный вес.
Каждую пару вершин выводите в отдельной строке. В каждой паре сначала выводится вершина с меньшим номером.
Нумерация вершин начинается с единицы.

Пары, отвечающие за рёбра имеющие максимальный вес, должны быть отсортированы в порядке возрастания.
Пары, отвечающие за рёбра имеющие минимальный вес, должны быть отсортированы в порядке убывания.
Пары вершин сравниваются так: 
$$ (i_1, j_1) > (i_2, j_2) \Leftrightarrow (i_1 > i_2) \ or \ ((i_1=i_2) \ and \ (j_1>j_2))$$

\Example

\begin{example}
\exmp{5
0 2 0 5 5 
2 0 3 0 0 
0 3 0 2 2 
5 0 2 0 4 
5 0 2 4 0 
}{
1 4
1 5
3 5
3 4
1 2
}%
\end{example}

\end{problem}

