\begin{problem}{Наивный алгоритм (первое вхождение)}{}{}{0.2 секунды}{64 мегабайта}

Даны строки $P$ и $T$. Найдите первое вхождение строки $P$ в текст $T$, используя {\bf наивный алгоритм поиска}, 
при котором выполняется посимвольное сравнение $P$ с каждой из подстрок $T$ длины $|P|$. 

Подстроки $T$ перебираются слева направо, символы строк также сравниваются слева направо. 
При каждом сравнении символов (независимо от успешности сравнения) надо выводить сравниваемый символ строки $P$. 
После завершения алгоритма надо вывести позицию в $T$, с которой начинается первое вхождение $P$, или $0$, если вхождения отсутствуют. 

\InputFile
В первой строке входных данных записан образец $P$, 
во второй строке записан текст $T$ ($1 \le |P| \le 100$, $1 \le |T| \le 100$).


\OutputFile
В первую строку выведите сравниваемые символы образца $P$.
Во вторую строку выведите позицию первого вхождения образца $P$ в текст $T$ или $0$, если вхождения отсутствуют.


\Examples

\begin{example}
\exmp{abbbbbabb
aaabbbbbabbababbabbbabbab
}{abababbbbbabb
3}%
\end{example}

\end{problem}
