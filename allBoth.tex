\documentclass [10pt, a4paper, oneside] {article}


\usepackage{cmap}
\usepackage [T2A] {fontenc}
\usepackage [utf8] {inputenc}
\usepackage [english, russian] {babel}
\usepackage {amsmath}
\usepackage {amssymb}
\usepackage [ukrainian, arabic]{olymp}
\usepackage {scoring}
\usepackage {comment}
\usepackage {epigraph}
\usepackage {expdlist}
\usepackage{graphicx}
\usepackage{wrapfig}
\usepackage{color}
\usepackage{listings}
\usepackage{import}

\usepackage{ifpdf}
\ifpdf
  \DeclareGraphicsRule{*}{mps}{*}{}
\fi


%\usepackage{hyperref}

\begin {document}
%----------------
\definecolor{mygreen}{rgb}{0,0.8,0}
\definecolor{mygray}{rgb}{0.5,0.5,0.5}
\definecolor{mymauve}{rgb}{0.58,0,0.82}
\definecolor{myyellow}{rgb}{1,0.34,0}
\lstset{%
language=Pascal,                 % выбор языка для подсветки 
extendedchars=true,
inputencoding=koi8-r,
%basicstyle=\small\sffamily, % размер и начертание шрифта для подсветки кода
basicstyle=\small\ttfamily, % размер и начертание шрифта для подсветки кода
classoffset=1,
morekeywords={div,mod,input,output,var,const,type,begin,end,if,then,else,while,for,do,repeat,until},keywordstyle=\color{myyellow},
classoffset=2,
morekeywords={integer,longint,real,word,qword,int64,array},keywordstyle=\color{mygreen},
classoffset=0,
deletekeywords={div,mod,input,output,var,const,type,begin,end,if,then,else,while,for,do,repeat,until},
deletekeywords={integer,longint,real,word,qword,int64,array},
keywordstyle=\color{blue},
identifierstyle=\color{magenta},
commentstyle=\color{black},
numbers=left,               % где поставить нумерацию строк (слева\справа)
numberstyle=\tiny,           % размер шрифта для номеров строк
stepnumber=1,                   % размер шага между двумя номерами строк
numbersep=5pt,                % как далеко отстоят номера строк от подсвечиваемого кода
backgroundcolor=\color{white}, % цвет фона подсветки - используем \usepackage{color}
showspaces=false,            % показывать или нет пробелы специальными отступами
showstringspaces=false,      % показывать или нет пробелы в строках
showtabs=false,             % показывать или нет табуляцию в строках
frame=single,              % рисовать рамку вокруг кода
tabsize=2,                 % размер табуляции по умолчанию равен 2 пробелам
captionpos=t,              % позиция заголовка вверху [t] или внизу [b] 
breaklines=true,           % автоматически переносить строки (да\нет)
%breakatwhitespace=false, % переносить строки только если есть пробел
escapeinside={\%*}{*)}  % если нужно добавить комментарии в коде
}


	

\contest
{PutContestLongNameHere}%
{}%
{}%

\binoppenalty=10000
\relpenalty=10000
\setcounter{problem}{0}

\renewcommand{\t}{\texttt}

\def\rururu#1{
\graphicspath{{#1/statements/}}
\import{#1/statements/}{./problem_ru_pdf.tex}}

\def\ukukuk#1{
\graphicspath{{#1/statements/}}
\import{#1/statements/}{./problem_uk_pdf.tex}}


\rururu{2142}
\rururu{2143}
\rururu{2144}
\rururu{2145}
\rururu{2146}
\rururu{2147}
\rururu{2148}
\rururu{2149}
\rururu{2150}
\rururu{2151}
\rururu{2152}
\rururu{2153}
\rururu{2154}
\rururu{2155}
\rururu{2156}
\rururu{2157}
\rururu{2158}
\rururu{2159}
\rururu{2160}
\rururu{2161}
\rururu{2162}
\rururu{2163}
\rururu{2164}
\rururu{2165}
\rururu{2166}

\setcounter{problem}{0}


\ukukuk{2142}
\ukukuk{2143}
\ukukuk{2144}
\ukukuk{2145}
\ukukuk{2146}
\ukukuk{2147}
\ukukuk{2148}
\ukukuk{2149}
\ukukuk{2150}
\ukukuk{2151}
\ukukuk{2152}
\ukukuk{2153}
\ukukuk{2154}
\ukukuk{2155}
\ukukuk{2156}
\ukukuk{2157}
\ukukuk{2158}
\ukukuk{2159}
\ukukuk{2160}
\ukukuk{2161}
\ukukuk{2162}
\ukukuk{2163}
\ukukuk{2164}
\ukukuk{2165}
\ukukuk{2166}









\end {document}
